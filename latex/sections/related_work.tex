% Related Work (Literature Review)
\section{Related Work}

\subsection{Energy-Efficient Clustering in WSNs}
Hierarchical clustering protocols have been central to WSN energy optimization since the seminal work of Heinzelman et al.~\cite{heinzelman2000leach}. LEACH introduced randomized, probabilistic cluster head (CH) rotation, achieving distributed load balancing but with limited adaptivity to residual energy or spatial heterogeneity. Younis and Fahmy~\cite{younis2004heed} proposed HEED, which extends LEACH by incorporating residual energy and intra-cluster communication cost into CH selection, yielding more stable clusters. Smaragdakis et al.~\cite{smaragdakis2004sep} addressed heterogeneous networks via SEP, weighting CH election probabilities by initial energy tier, thus prolonging network lifetime through preferential selection of advanced nodes.

[EXPANDED] The DEEC family (Distributed Energy-Efficient Clustering)~\cite{qing2006deec} further refines heterogeneous clustering by dynamically adjusting CH election probabilities based on the ratio of residual-to-initial energy, enabling nodes with higher remaining capacity to serve as CHs more frequently. Variants include DDEEC (Developed DEEC)~\cite{elbhiri2010ddeec}, which introduces three energy levels (normal, advanced, super), and EDEEC (Enhanced DEEC)~\cite{saini2010edeec}, which adapts cluster formation thresholds per round. However, all DEEC-family protocols optimize purely for energy balance and do not account for \emph{post-transmission cooling intervals}—the enforced temporal gap required after high-power transmissions to prevent thermal damage or regulatory violations. As a result, rapid successive CH re-elections or routing through recently active nodes can inadvertently degrade effective throughput and introduce hidden latency.

\subsection{Cooling and Thermal Constraints in Wireless Systems}
Thermal management in wireless devices has been explored primarily in cellular base stations~\cite{thermal_cellular2018} and high-density IoT deployments~\cite{iot_thermal2020}, where processor throttling and transmission scheduling are adjusted to maintain safe operating temperatures. However, explicit cooling-state modeling in WSN routing and clustering remains nascent. Recent work by Zhang et al.~\cite{zhang_cooling2021} introduced cooling-aware duty cycling for individual nodes but did not integrate cooling into multi-hop routing or CH selection cost functions. Our approach differs by treating cooling state as a \emph{first-class optimization variable} across all three subsystems: clustering, routing, and coverage control.

\subsection{Coverage Optimization and Redundancy Management}
Coverage maximization under energy constraints has been addressed through sleep scheduling~\cite{ye2003peas,tian2002node} and adaptive sensing radius control~\cite{wang2007coverage}. PEAS~\cite{ye2003peas} uses probing to keep only necessary sensors active, while Tian and Georganas~\cite{tian2002node} proposed node scheduling based on geometric coverage overlap. However, these methods do not consider the interplay between redundancy, energy expenditure, and cooling overhead. Our \emph{redundancy-driven sleep--wake optimization with adaptive radius contraction} uniquely couples spatial redundancy (unique coverage $U_i$) with energy and cooling state to co-optimize coverage preservation and thermal sustainability.

\subsection{Routing Under Operational Constraints}
Directed diffusion~\cite{intanagonwiwat2000directed} and gradient-based routing~\cite{schurgers2002energy} optimize paths based on energy reserves and link quality, but do not account for transient node unavailability due to cooling. [ADDED] Thermal-aware routing has emerged in recent years: TADR (Thermal-Aware Delay-constrained Routing)~\cite{chen2019tadr} minimizes end-to-end delay by selecting paths with lower cumulative thermal load, measured via temperature sensors at each node. However, TADR does not model explicit cooling \emph{state transitions} (i.e., mandatory rest periods) and instead uses instantaneous temperature as a continuous metric. Our cooling-aware Dijkstra variant differs by: (i) treating cooling as a discrete state variable ($C_i \in \{0,1,\ldots,\text{MinRest}\}$), (ii) imposing deterministic edge penalties ($\lambda_{\text{cool}} \cdot C_i/\text{MinRest}$) that enforce path avoidance during mandatory rest, and (iii) integrating this routing logic with cluster-head selection and sleep--wake mechanisms, creating a unified cross-layer optimization framework. Recent energy-aware routing surveys~\cite{mittal2017survey} focus on residual battery levels and hop counts; our approach extends this by avoiding latent forwarding bottlenecks caused by cooling-induced node unavailability.

\subsection{Positioning This Work}
\Cref{tab:related-comparison} summarizes key distinctions. Our integrated framework is the first to:
\begin{enumerate}[label=(\roman*)]
  \item Incorporate explicit cooling cost ($\delta \frac{C_i}{\text{MinRest}}$) into CH selection alongside distance, energy, and density.
  \item Penalize routing edges originating at cooling nodes via $\lambda_{cool}$ surcharges, ensuring paths avoid thermal bottlenecks.
  \item Unify redundancy-based sleep scheduling with adaptive sensing-radius contraction, preserving macro coverage while reducing per-node load and cooling frequency.
\end{enumerate}

\begin{table}[ht]
  \centering
  \caption{Comparison with related WSN optimization approaches.}
  \label{tab:related-comparison}
  \begin{tabular}{@{}lccccc@{}}
    \toprule
    Method & Clustering & Routing & Coverage & Cooling & Heterogeneity \\
    \midrule
    LEACH~\cite{heinzelman2000leach} & Random & CH-direct & Fixed & No & Homogeneous \\
    HEED~\cite{younis2004heed} & Energy+cost & CH-direct & Fixed & No & Homogeneous \\
    SEP~\cite{smaragdakis2004sep} & Weighted prob. & CH-direct & Fixed & No & 2-tier energy \\
    DEEC~\cite{qing2006deec} & Dynamic prob. & CH-direct & Fixed & No & 2-tier energy \\
    EDEEC~\cite{saini2010edeec} & Adaptive thresh. & CH-direct & Fixed & No & 3-tier energy \\
    TADR~\cite{chen2019tadr} & -- & Thermal-aware & Fixed & Temp-based & Homogeneous \\
    PEAS~\cite{ye2003peas} & -- & -- & Probing-based & No & Homogeneous \\
    Zhang et al.~\cite{zhang_cooling2021} & -- & -- & Duty-cycle & Node-level & Homogeneous \\
    \textbf{Proposed} & \textbf{Multi-factor} & \textbf{Cooling-aware} & \textbf{Redundancy+adaptive} & \textbf{Integrated} & \textbf{2-tier regional} \\
    \bottomrule
  \end{tabular}
\end{table}

This integrated cooling-aware paradigm yields substantial empirical gains (detailed in \Cref{sec:results}) while addressing a gap in the literature: the holistic incorporation of thermal state constraints into WSN optimization.
