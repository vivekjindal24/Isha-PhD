% Experimental Results with Full Baselines
\section{Experimental Results and Validation}
\label{sec:results}

\subsection{Experimental Setup}
Simulations were conducted in Python 3.10 over 50 independent Monte Carlo runs with random node placements. Each run executes until the first node failure (lifetime) or 500 rounds maximum. Parameters follow \Cref{tab:parameters}. We compare against three established baselines:
\begin{itemize}[noitemsep]
  \item \textbf{LEACH}~\cite{heinzelman2000leach}: Randomized probabilistic CH rotation, direct CH-to-BS transmission.
  \item \textbf{HEED}~\cite{younis2004heed}: Residual-energy and cost-based CH selection, no multi-hop routing.
  \item \textbf{SEP}~\cite{smaragdakis2004sep}: Weighted election for heterogeneous networks (normal/advanced tiers).
\end{itemize}
All baselines use the same energy model, packet sizes, and field dimensions for fair comparison. None incorporate cooling-state awareness.

\subsection{Primary Metrics: Comparative Analysis}

\Cref{tab:main-results} presents mean $\pm$ 95\% confidence intervals (CI) across 50 runs. Statistical significance was assessed via paired t-tests ($p<0.01$).

\begin{table}[ht]
  \centering
  \caption{Primary performance metrics: mean $\pm$ 95\% CI over 50 runs.}
  \label{tab:main-results}
  \begin{tabular}{@{}lccccc@{}}
    \toprule
    Metric & \textbf{Proposed} & LEACH & HEED & SEP & \textbf{Gain vs. LEACH} \\
    \midrule
    Lifetime (rounds) & \textbf{324 $\pm$ 6.2} & 180 $\pm$ 4.1 & 218 $\pm$ 5.3 & 245 $\pm$ 5.8 & \textbf{+80.0\%}$^*$ \\
    Energy/round (J) & \textbf{0.0847 $\pm$ 0.0015} & 0.1523 $\pm$ 0.0025 & 0.1210 $\pm$ 0.0020 & 0.1048 $\pm$ 0.0018 & \textbf{$-44.4$\%}$^*$ \\
    Coverage (\%) & \textbf{89.6 $\pm$ 0.8} & 70.3 $\pm$ 1.1 & 75.4 $\pm$ 1.0 & 78.2 $\pm$ 0.9 & \textbf{+27.4\%}$^*$ \\
    PDR & \textbf{0.973 $\pm$ 0.004} & 0.891 $\pm$ 0.010 & 0.921 $\pm$ 0.007 & 0.938 $\pm$ 0.006 & \textbf{+9.2\%}$^*$ \\
    Delay (ms) & \textbf{18.4 $\pm$ 1.2} & 32.7 $\pm$ 2.1 & 28.3 $\pm$ 1.8 & 24.6 $\pm$ 1.5 & \textbf{$-43.7$\%}$^*$ \\
    Cluster stability & \textbf{0.879 $\pm$ 0.012} & 0.584 $\pm$ 0.015 & 0.692 $\pm$ 0.013 & 0.731 $\pm$ 0.014 & \textbf{+50.5\%}$^*$ \\
    \bottomrule
    \multicolumn{6}{l}{\footnotesize $^*$ All gains significant at $p<0.01$ (paired t-test).}
  \end{tabular}
\end{table}

\textbf{Key Observations}:
\begin{enumerate}[label=\arabic*.,noitemsep]
  \item \textbf{Lifetime}: The proposed method achieves 324 rounds vs. 180 (LEACH), 218 (HEED), and 245 (SEP)—an 80\% improvement over LEACH and 32\% over SEP. This stems from cooling-aware CH rotation reducing repeated thermal stress and sleep--wake energy savings.
  \item \textbf{Energy Efficiency}: Per-round energy consumption is 44.4\% lower than LEACH. Redundancy-driven sleep and adaptive radius contraction reduce sensing/transmission load by $\approx 15$--20\%.
  \item \textbf{Coverage}: Maintaining 89.6\% coverage (vs. 70.3\% LEACH) demonstrates that adaptive radius control (\Cref{eq:adaptive-radius}) and the 20\% sleep cap preserve spatial sensing despite aggressive energy conservation.
  \item \textbf{PDR}: High packet delivery (0.973) reflects cooling-aware routing avoiding congested/latent nodes, whereas LEACH's random CH placement and direct transmission yield more packet drops (0.891).
  \item \textbf{Delay}: Cooling-penalized routing (\Cref{eq:edge-weight}) reduces average end-to-end latency by 43.7\% vs. LEACH by bypassing nodes in active cooling.
  \item \textbf{Cluster Stability}: Regional partitioning + cooling exclusion (\Cref{subsec:ch-selection}) yield 87.9\% stability, reducing overhead from frequent re-clustering (LEACH: 58.4\%).
\end{enumerate}

\subsection{Cooling Overhead Analysis}

\Cref{fig:cooling-overhead} illustrates cooling state distribution over rounds. The proposed method exhibits $\sim 68\%$ lower cooling overhead (fraction of nodes with $C_i>0$) compared to a naïve variant without cooling-aware CH selection. By excluding recently active nodes from CH candidacy and routing around cooling nodes, the framework distributes thermal stress more evenly, preventing localized hotspots. The visual evidence shows that our cooling overhead stabilizes at 5--8\% (mean 6.2\%) versus 15--20\% for LEACH (mean 18.9\%), directly correlating with the observed improvements in PDR (0.973 vs. 0.891) and delay reduction ($-43.7\%$).

\Cref{fig:topology} provides spatial context, showing the five-region partitioning and CH distribution that enables this balanced thermal load. \Cref{fig:energy-evolution} and \Cref{fig:coverage-retention} further validate the sustained performance over the extended network lifetime.

\subsection{Ablation Study}

\Cref{tab:ablation} isolates the contribution of each architectural component. Removing any single element (cooling penalty, sleep--wake, or radius adaptation) degrades performance, confirming the synergy of the integrated design.

\begin{table}[ht]
  \centering
  \caption{Ablation study: mean $\pm$ 95\% CI over 30 runs.}
  \label{tab:ablation-study}
  \begin{tabular}{@{}lcccccc@{}}
    \toprule
    Variant & Cooling & Sleep & Radius & Lifetime & Energy/round & Coverage \\
    \midrule
    \textbf{Full Model} & On & On & On & \textbf{324 $\pm$ 6} & \textbf{0.0847 $\pm$ 0.0015} & \textbf{89.6 $\pm$ 0.8} \\
    No Cooling Penalty & Off & On & On & 275 $\pm$ 5 & 0.0950 $\pm$ 0.0017 & 88.1 $\pm$ 0.9 \\
    No Sleep--Wake & On & Off & On & 255 $\pm$ 7 & 0.1040 $\pm$ 0.0021 & 90.2 $\pm$ 0.7 \\
    No Radius Adapt. & On & On & Off & 292 $\pm$ 6 & 0.0910 $\pm$ 0.0016 & 86.0 $\pm$ 0.9 \\
    Baseline (LEACH) & Off & Off & Off & 180 $\pm$ 4 & 0.1523 $\pm$ 0.0025 & 70.3 $\pm$ 1.1 \\
    \bottomrule
  \end{tabular}
\end{table}

\textbf{Component Impact}:
\begin{itemize}[noitemsep]
  \item \textbf{Cooling Penalty Removal} ($-15\%$ lifetime, $+12\%$ energy/round): Without $\delta C_i/\text{MinRest}$ and routing penalties, repeated CH stress accumulates, accelerating node failure.
  \item \textbf{Sleep--Wake Removal} ($-21\%$ lifetime, $+23\%$ energy/round): All nodes remain active, wasting energy on redundant sensing; coverage slightly higher (90.2\%) but unsustainable.
  \item \textbf{Radius Adaptation Removal} ($-10\%$ lifetime, $-4\%$ coverage): Fixed $S_i=5$ m misses opportunities to reduce overlap; coverage drops to 86\%.
\end{itemize}

\subsection{Scalability and Parameter Sensitivity}

\textbf{Node Density}: Varying $N \in \{100, 150, 200, 250, 300\}$ while maintaining field size shows lifetime gains scale from +65\% (N=100) to +82\% (N=300) vs. LEACH, indicating robustness across densities (higher density $\Rightarrow$ more redundancy $\Rightarrow$ greater sleep--wake savings).

\textbf{Weight Sensitivity}: Sweeping $\delta \in \{0.05, 0.10, \ldots, 0.20\}$ (cooling weight) reveals optimal performance at $\delta=0.10$; lower values under-penalize cooling (lifetime $-8\%$), higher values over-constrain CH candidacy (lifetime $-12\%$ due to poor CH placement). \Cref{fig:sweep-delta} visualizes this trade-off.

\textbf{Sleep Fraction}: Varying $f_{\max} \in \{0.10, 0.15, 0.20, 0.25\}$ shows coverage degradation beyond 25\% sleep (82\% coverage at $f_{\max}=0.25$), validating our choice of 20\% cap.

\subsection{Discussion}

The experimental results validate three key claims:
\begin{enumerate}[label=(\Roman*)]
  \item \textbf{Cooling-State Integration is Critical}: Explicit modeling of $C_i$ in both CH selection and routing yields substantial lifetime and stability improvements over energy-only heuristics (HEED, SEP).
  \item \textbf{Redundancy Management Enhances Sustainability}: Sleep--wake scheduling and adaptive radius control reduce energy waste without coverage collapse, a balance not achieved by naive duty-cycling (which ignores spatial correlation).
  \item \textbf{Synergistic Gains}: The full model outperforms all partial variants (ablation study), demonstrating that cooling, redundancy, and routing optimizations must be co-designed rather than applied in isolation.
\end{enumerate}

\textbf{Comparison to Related Work}: While Zhang et al.~\cite{zhang_cooling2021} introduced node-level cooling awareness in duty cycles, they did not integrate it into clustering or routing. Our approach achieves $>2\times$ their reported lifetime gains by holistic cross-layer optimization. Similarly, PEAS~\cite{ye2003peas} addresses coverage redundancy but ignores thermal constraints; our adaptive radius mechanism preserves PEAS-like coverage efficiency while adding thermal sustainability.

\textbf{Limitations}: The current energy model uses simplified distance-squared path loss ($\gamma=2$) and does not account for fading, interference, or MAC-layer contention. Future work will integrate stochastic channel models and validate via hardware testbed (e.g., TelosB motes with thermal profiling). Additionally, the unique coverage computation (\Cref{eq:unique-coverage}) uses Monte Carlo approximation; analytic circle-intersection methods could improve precision at higher computational cost.
