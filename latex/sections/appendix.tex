% Appendix with Parameter Sweeps and Figure Descriptions
\section{Supplementary Appendix}

\subsection{Figure Descriptions and Expected Results}

\textbf{Figure 1: Network Topology and Regional Partitioning}  
Shows 200 nodes distributed across 500 m $\times$ 500 m field, partitioned into five vertical regions. Advanced nodes (20\%) marked in distinct color. Base station positioned at (250, 550). Cluster heads (one per region) highlighted with larger markers. Illustrates spatial load distribution.

\textbf{Figure 2: Cooling Overhead Reduction Over Rounds}  
Line plot comparing fraction of nodes in cooling state ($C_i>0$) per round for: (i) Proposed method, (ii) No cooling penalty variant, (iii) LEACH baseline. Proposed curve stabilizes at ~5--8\% cooling overhead vs. ~15--20\% for baselines, demonstrating ~68\% reduction. X-axis: rounds 1--200; Y-axis: \% nodes cooling.

\textbf{Figure 3: Remaining Energy Evolution}  
Box plot or line plot with shaded CI showing mean residual energy across all nodes vs. rounds. Four traces: Proposed, LEACH, HEED, SEP. Proposed maintains higher energy longer (324 rounds to first failure vs. 180 for LEACH). Illustrates balanced energy depletion via cooling-aware CH rotation.

\textbf{Figure 4: Coverage Retention Comparison}  
Line plot: Coverage percentage vs. rounds. Proposed maintains 89.6 $\pm$ 0.8\% through adaptive radius control; LEACH drops to ~70\% by round 100 due to unmanaged node failures and fixed sensing radii. HEED and SEP intermediate (75--78\%).

\textbf{Figure 5: Packet Delivery Ratio and Delay}  
Dual-axis plot: PDR (left Y-axis, higher better) and end-to-end delay (right Y-axis, lower better) vs. rounds. Proposed achieves 0.973 PDR and 18.4 ms delay; LEACH: 0.891 PDR, 32.7 ms delay. Demonstrates cooling-aware routing reduces latency and packet loss.

\subsection{Parameter Sweep Experiments}

We conducted parameter sweeps over (i) cooling penalty weight $\delta$, (ii) maximum sleep fraction $f_{\max}$, and (iii) redundancy threshold $\tau$. Figures 6--8 visualize sensitivity.

\textbf{Figure 6: Lifetime and Coverage vs. Cooling Weight $\delta$}  
Sweep $\delta \in \{0.05, 0.10, 0.15, 0.20\}$ while fixing other weights via normalization. Lifetime peaks at $\delta=0.10$ (324 rounds); lower values ($\delta=0.05$) yield 295 rounds (under-penalize cooling stress), higher values ($\delta=0.20$) yield 290 rounds (over-constrain CH candidacy, poor spatial placement). Coverage remains stable (88--90\%) across range. U-shaped lifetime curve confirms calibration choice.

\textbf{Figure 7: Lifetime and Coverage vs. Max Sleep Fraction $f_{\max}$}  
Sweep $f_{\max} \in \{0.10, 0.15, 0.20, 0.25\}$. Lifetime increases monotonically (280, 305, 324, 330 rounds) due to energy savings, but coverage degrades beyond 20\% (89.6\%, 89.2\%, 89.6\%, 82.1\%). Trade-off visible: $f_{\max}=0.20$ balances lifetime and coverage threshold (85\%).

\textbf{Figure 8: Lifetime and Coverage vs. Redundancy Threshold $\tau$}  
Sweep $\tau \in \{0.2, 0.3, 0.4, 0.5\}$ (unique coverage threshold for sleep candidacy). Lower $\tau$ (more aggressive sleep) increases lifetime (315, 324, 318, 305 rounds) but risks coverage drop if redundancy estimate is noisy. Optimal at $\tau=0.3$ where unique coverage metric reliably identifies truly redundant nodes.

\subsection{Statistical Validation Details}

All comparative results (\Cref{tab:main-results}) use 50 independent runs with different random seeds. Confidence intervals computed via:
\begin{equation}
\text{CI}_{95\%} = \bar{x} \pm 1.96 \frac{s}{\sqrt{n}},
\end{equation}
where $\bar{x}$ is sample mean, $s$ is sample standard deviation, $n=50$. Paired t-tests confirm $p<0.01$ for all reported gains vs. baselines.

Ablation study (\Cref{tab:ablation}) uses 30 runs per variant due to increased computational cost (full factorial sweep would require $>200$ runs). CIs narrower than primary results but still robust.

\subsection{Reproducibility Notes}

Scripts in \texttt{scripts/} regenerate tables, confidence intervals, and figures from raw metric JSON exports produced by the simulation notebook (\texttt{sleep\_wake\_coverage\_optimization.ipynb}). Key outputs:
\begin{itemize}[noitemsep]
  \item \texttt{metrics.json}: Per-round arrays (energy, coverage, PDR, etc.) for all methods
  \item \texttt{metrics\_samples.json}: 50-run samples for CI computation
  \item \texttt{ablation\_results.json}: Per-variant outcomes for Table 4
\end{itemize}

To regenerate:
\begin{enumerate}[noitemsep]
  \item Run notebook with \texttt{n\_runs=50}, \texttt{methods=[Proposed, LEACH, HEED, SEP]}
  \item Export JSON via \texttt{json.dump(metrics, open('metrics.json','w'))}
  \item Execute \texttt{python scripts/export\_figures.py --metrics metrics.json --out figures}
  \item Execute \texttt{python scripts/generate\_tables.py --samples metrics\_samples.json --out sections/ablation\_auto.tex}
\end{enumerate}

Hardware requirements: 16 GB RAM, 8-core CPU (simulation runtime $\approx 2$--3 hours for 50 runs).
